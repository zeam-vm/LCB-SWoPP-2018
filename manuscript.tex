\documentclass[submit,techrep]{ipsj}


\usepackage{booktabs} % For formal tables
\usepackage[table,xcdraw]{xcolor}

\usepackage[dvipdfmx]{graphicx}
\usepackage{latexsym}
\usepackage{relsize}
\usepackage{xspace}
\usepackage{jtygm}

\usepackage{color}
\usepackage{fancyvrb}
\newcommand{\VerbBar}{|}
\newcommand{\VERB}{\Verb[commandchars=\\\{\}]}
\DefineVerbatimEnvironment{Highlighting}{Verbatim}{commandchars=\\\{\}}
% Add ',fontsize=\small' for more characters per line
\newenvironment{Shaded}{}{}
\newcommand{\AlertTok}[1]{\textcolor[rgb]{1.00,0.00,0.00}{\textbf{#1}}}
\newcommand{\AnnotationTok}[1]{\textcolor[rgb]{0.38,0.63,0.69}{\textbf{\textit{#1}}}}
\newcommand{\AttributeTok}[1]{\textcolor[rgb]{0.49,0.56,0.16}{#1}}
\newcommand{\BaseNTok}[1]{\textcolor[rgb]{0.25,0.63,0.44}{#1}}
\newcommand{\BuiltInTok}[1]{#1}
\newcommand{\CharTok}[1]{\textcolor[rgb]{0.25,0.44,0.63}{#1}}
\newcommand{\CommentTok}[1]{\textcolor[rgb]{0.38,0.63,0.69}{\textit{#1}}}
\newcommand{\CommentVarTok}[1]{\textcolor[rgb]{0.38,0.63,0.69}{\textbf{\textit{#1}}}}
\newcommand{\ConstantTok}[1]{\textcolor[rgb]{0.53,0.00,0.00}{#1}}
\newcommand{\ControlFlowTok}[1]{\textcolor[rgb]{0.00,0.44,0.13}{\textbf{#1}}}
\newcommand{\DataTypeTok}[1]{\textcolor[rgb]{0.56,0.13,0.00}{#1}}
\newcommand{\DecValTok}[1]{\textcolor[rgb]{0.25,0.63,0.44}{#1}}
\newcommand{\DocumentationTok}[1]{\textcolor[rgb]{0.73,0.13,0.13}{\textit{#1}}}
\newcommand{\ErrorTok}[1]{\textcolor[rgb]{1.00,0.00,0.00}{\textbf{#1}}}
\newcommand{\ExtensionTok}[1]{#1}
\newcommand{\FloatTok}[1]{\textcolor[rgb]{0.25,0.63,0.44}{#1}}
\newcommand{\FunctionTok}[1]{\textcolor[rgb]{0.02,0.16,0.49}{#1}}
\newcommand{\ImportTok}[1]{#1}
\newcommand{\InformationTok}[1]{\textcolor[rgb]{0.38,0.63,0.69}{\textbf{\textit{#1}}}}
\newcommand{\KeywordTok}[1]{\textcolor[rgb]{0.00,0.44,0.13}{\textbf{#1}}}
\newcommand{\NormalTok}[1]{#1}
\newcommand{\OperatorTok}[1]{\textcolor[rgb]{0.40,0.40,0.40}{#1}}
\newcommand{\OtherTok}[1]{\textcolor[rgb]{0.00,0.44,0.13}{#1}}
\newcommand{\PreprocessorTok}[1]{\textcolor[rgb]{0.74,0.48,0.00}{#1}}
\newcommand{\RegionMarkerTok}[1]{#1}
\newcommand{\SpecialCharTok}[1]{\textcolor[rgb]{0.25,0.44,0.63}{#1}}
\newcommand{\SpecialStringTok}[1]{\textcolor[rgb]{0.73,0.40,0.53}{#1}}
\newcommand{\StringTok}[1]{\textcolor[rgb]{0.25,0.44,0.63}{#1}}
\newcommand{\VariableTok}[1]{\textcolor[rgb]{0.10,0.09,0.49}{#1}}
\newcommand{\VerbatimStringTok}[1]{\textcolor[rgb]{0.25,0.44,0.63}{#1}}
\newcommand{\WarningTok}[1]{\textcolor[rgb]{0.38,0.63,0.69}{\textbf{\textit{#1}}}}


\def\newblock{\hskip .11em plus .33em minus .07em}

\def\Underline{\setbox0\hbox\bgroup\let\\\endUnderline}
\def\endUnderline{\vphantom{y}\egroup\smash{\underline{\box0}}\\}
\def\|{\verb|}

\newcommand{\Rplus}{\protect\hspace{-.1em}\protect\raisebox{.8ex}{\smaller{\smaller\smaller\textbf{+}}}}

\newcommand{\Cpp}{\mbox{C\Rplus\Rplus}\xspace}



\setcounter{巻数}{59}%vol53=2012
\setcounter{号数}{0}
\setcounter{page}{0}


\def\tightlist{\itemsep1pt\parskip0pt\parsep0pt}

\begin{document}
\title{Nodeプログラミングモデルを活用した \Cpp およびElixirの実行環境の実装}
\etitle{Implementation of Runtime Environments of  \Cpp  and Elixir with the Node Programming Model}

\paffiliate{Kitakyu-u}{北九州市立大学\\
University of Kitakyushu}
\affiliate{Delight}{有限会社デライトシステムズ\\
Delight Systems Co., Ltd.}
\paffiliate{Kyoto-u}{京都大学\\
Kyoto University}


\author{山崎 進}{Yamazaki Susumu}{Kitakyu-u}[zacky@kitakyu-u.ac.jp]
\author{森 正和}{Mori Masakazu}{Delight}[mori@delightsystems.com]
\author{上野 嘉大}{Ueno Yoshihiro}{Delight}[delightadmin@delightsystems.com]
\author{高瀬 英希}{TAKASE Hideki}{Kyoto-u}[takase@i.kyoto-u.ac.jp]

\begin{abstract}
Node.jsでは,コールバックを用いてI/Oを非同期的に扱ってノンプリエンプティブなマルチタスクにする機構Nodeプログラミングモデルが備わっている.これにより,ウェブサーバーのメモリ使用量を格段に減らすことができ,同時セッション最大数やレイテンシが改善される.我々は \Cpp とElixirで同様の機構を実装した. \Cpp への実装をZackernelと称し,Elixirへの実装を軽量コールバックスレッドと称している.ZackernelはRFIDのような極端に小規模で消費電力の少ないIoTシステムを組む場合のカーネルとしての用途,軽量コールバックスレッドはクラウドサーバーでの用途をそれぞれ想定している.Zackernelは \Cpp 11で採用された匿名関数を利用して,dispatchメソッドにて次に呼び出すべき関数をキューから読み込んで呼び出すという原理で実現する.軽量コールバックスレッドは,Elixirでは関数が一級オブジェクトであるので,関数のリストをキューとして保持し,キューの先頭の関数の実行が終わったら次の関数を呼び出すという原理で実現する.今後,プロセス間通信の機能を実装し,ベンチマークプログラムに適用して性能を評価したい.
\end{abstract}


\begin{jkeyword}
Elixir, \Cpp, Node.js, マルチタスク
\end{jkeyword}


\begin{eabstract}
Node.js has the Node programming model, which is a non-preemptive multi-tasking mechanism that processes I/O asynchronously using callbacks. Thus, it improves maximum concurrent sessions and latency of web servers because it requires few memory. We implement it in \Cpp and Elixir. We call the implementation in \Cpp and Elixir Zackernel and light-weight callback threads (LCTs), respectively. We assume that Zackernel will be used as a kernel in an extremely small and power-saving IoT system such as an RFID-based IoT, and that LCTs will be used as a cloud server. Zackernel is implemented using anonymous functions, which has been adopted by \Cpp 11, by the principle that the {\tt dispatch} method calls back the next process function, with reading from the scheduler queue. LCTs is realized by the priciple that implements the scheduler queue as a function list and call back the next process function after running the function of the head of the queue because Elixir treats a function as a first-class object. We will implement an inter-process communication mechanism, apply it a benchmark, and evaluate performance of it.
\end{eabstract}

\begin{ekeyword}
Elixir,  \Cpp , Node.js, multi-tasking
\end{ekeyword}

\maketitle

\input{description}

%\begin{acknowledgment}
%\end{acknowledgment}


\bibliographystyle{ipsjsort}
\bibliography{reference}

\end{document}
